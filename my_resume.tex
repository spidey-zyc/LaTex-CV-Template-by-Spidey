% !TEX TS-program = xelatex
% !TEX encoding = UTF-8 Unicode
% !Mode:: "TeX:UTF-8"

\documentclass{resume}
\usepackage{zh_CN-Adobefonts_external} % 确保您有配套的字体文件夹
\usepackage{linespacing_fix} 
\usepackage{cite}
\usepackage{graphicx} % 必须引入,否则模板中的 \includegraphics 会报错

\begin{document}

\pagenumbering{gobble} % suppress displaying page number

% ==============================================================================
% 头部信息:照片、姓名、联系方式
% ==============================================================================



\name{张亦弛}

% 联系方式:模板定义只接受两个参数(手机、邮箱)
\contactInfo{(+86) XXX-XXXX-XXXX}{XXXXXXXXXX}

% 其他信息:利用模板的 \otherInfo 放置专业和主页,后两个参数留空
\otherInfo{人工智能(卓越人才试点班)}{https://spidey-zyc.github.io}{}{}

% 照片:根据模板逻辑,这里直接调用,图片需放在 images/you.jpg
% 参数 0.15 控制照片宽度占页面宽度的比例,可按需微调
\yourphoto{0.13}

% ==============================================================================
% 教育背景
% ==============================================================================
\section{教育背景}
\eduitem{上海交通大学}{计算机学院 · 人工智能(卓越班)}{\textbf{GPA}: 3.87/4.3 (前\textbf{30\%})}{2023.09 -- 至今}
\begin{itemize}

  \item \textbf{荣誉奖项}:上海交通大学本科生 C 等优秀奖学金 (2023-2024 学年);2024年全国大学生数学建模竞赛上海赛区二等奖
\end{itemize}



% ==============================================================================
% 项目经历
% ==============================================================================
\section{项目经历}
\datedsubsectionWithRole{\textbf{RAG 智能助教系统}}{\textbf{项目负责人}}{2025.11 -- 2025.12}
\begin{itemize}
\item \textbf{项目目标}:构建课程专用RAG系统,通过精准检索外部知识,解决大模型回答中的事实性“幻觉”问题。
\item \textbf{多模态索引}:利用 VLM 解析图表并与文本硬编码关联,构建统一向量库实现图文同步检索。图文关联召回率达 \textbf{85\%}。
  \item \textbf{检索调优}:融合 BM25 与语义检索并引入 Rerank 精排。Top-3 结果相关性从 75\% 提升至 \textbf{92\%},有效抑制生成幻觉。
  \item \textbf{项目成果}:将非结构化课件转化为高精度知识库。经量化验证,该方案显著增强了上下文相关性与答疑准确度。

\end{itemize}

\datedsubsectionWithRole{\textbf{GAKG-Explorer: 地学文献智能挖掘系统}}{\textbf{核心开发者}}{2025.09 -- 2026.01}
\begin{itemize}
  \item \textbf{SIS-Rank 图挖掘算法}:针对地学搜索噪声大的痛点,自研 Semantic-Inverse-Specificity 排序算法。通过“特异性稀释”抑制通用词干扰,配合“能量回流”提升专业长尾实体权重。在12组典型多义词对比测试中,系统综合相关性平均得分达\textbf{6.91/10},显著优于基线(\textbf{3.43/10})与仅向量检索方案(\textbf{2.04/10})。
  \item \textbf{多智能体协作架构}:构建基于 Qwen 模型及多智能体(Planner/Searcher/Writer)的流水线。通过 Sentence-Transformers 向量检索与图谱排序混合增强,实现从模糊查询到结构化深度综述的全自动生成。
  \item \textbf{工程落地}:使用 NetworkX 建模引文实体图谱,支持复杂地学概念(如“环流”、“页岩”)的精准语境消歧,有效打通了科研检索、阅读与综述分析的全流程。
\end{itemize}

\datedsubsectionWithRole{\textbf{深度强化学习算法研究与复现 (Atari/MuJoCo)}}{\textbf{独立开发者}}{2025.12 -- 2026.01}
\begin{itemize}
  \item \textbf{算法实现与对比}:在 Atari (Pong, Breakout) 与 MuJoCo (HalfCheetah, Ant) 环境下,复现并对比了 DQN, Double DQN (DDQN), PPO, TD3 等主流算法。
  \item \textbf{性能优化}:针对 TD3 算法在连续控制任务中的表现进行调参优化(学习率、探索噪声等)。实验结果显示,优化后的 TD3 在 HalfCheetah 任务中平均回报突破 \textbf{8000+},收敛速度与稳定性均优于 PPO。
  \item \textbf{消融实验}:设计实验探究 Epsilon 衰减策略及 Batch Size 对 DDQN 收敛性的影响,在 Pong 环境中实现接近满分(20+)的稳定表现。
\end{itemize}

\datedsubsectionWithRole{\textbf{基于 GAN 的人脸图像生成与优化}}{\textbf{核心开发者}}{2024.03 -- 2024.06}
\begin{itemize}
  \item \textbf{模型复现与改进}:基于 CelebA 数据集复现 DCGAN,针对模式崩溃 (Mode Collapse) 问题,引入 \textbf{Replay Buffer (经验回放池)} 机制与 \textbf{Spectral Normalization (谱归一化)},显著提升了长周期训练的稳定性,在长周期(\textbf{100+} epochs)下FID指标平稳在50、IS平稳在2.5左右。
  \item \textbf{隐空间分析}:利用 PCA 主成分分析与 t-SNE 算法对 Latent Space 进行降维可视化,验证了模型对性别、姿态等高层语义特征的解耦能力。
  \item \textbf{StyleGAN 对比研究}:复现 StyleGAN 模型并进行对比实验,评估 FID 与 IS 指标,验证了基于 Style 的生成架构在图像质量与多样性上的优势;整体而言,StyleGAN的FID优化了约\textbf{50\%},IS提升了约\textbf{40\%}。
\end{itemize}






% ==============================================================================
% 技术能力
% ==============================================================================
\section{技术能力}
\begin{itemize}
  \item \textbf{编程语言}: Python, C++, Shell
  \item \textbf{AI与数据栈}: PyTorch, NumPy, Pandas, Matplotlib, WandB, NetworkX
  \item \textbf{工具与文档}: LaTeX, Markdown, Git, Linux, Docker
  \item \textbf{英语能力}: 六级 571 分,托福 104 分,具备流利的英语听说读写能力
\end{itemize}

% ==============================================================================
% 竞赛获奖

% 



% 学生工作与社会实践
% ==============================================================================
\section{学生工作与社会实践}

\datedsubsectionWithRole{\textbf{上海交通大学“荣昶储才计划” (第九期学员)}}{\textbf{骨干学员}}{2023.09 -- 至今}
\role{行政组第四组、健康组、文化组成员}{选拔率 < 50人/年}
\begin{itemize}
  \item \textbf{核心培养}:入选交大优秀学生培养计划,通过“聚焦基层、认知行业、国际视野”三阶段社会实践,提升领导力。
  \item \textbf{组织策划}:作为健康组与文化组成员,主导策划\textbf{4}次体育赛事与文化沙龙活动;参与“学子讲堂”与读书会,撰写\textbf{3}篇深度社会观察报告。
\end{itemize}

\datedsubsectionWithRole{\textbf{上海交通大学 健步足球协会}}{\textbf{会长}}{2025.09 -- 至今}
\begin{itemize}
  \item 统筹社团日常运营,对接校体育系场地资源,组织校内足球赛事每学期\textbf{18}次,提升社团校内影响力,招新\textbf{100}余人。
\end{itemize}

\datedsubsectionWithRole{\textbf{上海交通大学 魔方协会}}{\textbf{副会长}}{2023.12 -- 2024.09}
\begin{itemize}
  \item 负责魔方竞速赛事的流程设计与现场执行,成功开展2024高校魔方赛并获取活动赞助。
\end{itemize}

\datedsubsectionWithRole{\textbf{个人自媒体运营 “一池安易持”}}{\textbf{主理人}}{2023.09 -- 至今}
\begin{itemize}
  \item 运营个人公众号,累计粉丝 \textbf{2000+}。撰写并发布 AI 技术科普、大学生活经验及深度思考类文章,最高阅读量\textbf{50000+}。
\end{itemize}

\end{document}